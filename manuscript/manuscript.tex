\documentclass{article}

\usepackage{color}
\usepackage{listings}

\begin{document}

\section*{Methods}
\subsection*{Data collection}
We have obtained monthly water temperature readings from 1978 through 2015 via the Washington Department of Ecology's River and Stream Water Quality Monitoring program \colorbox{red}{\lstinline{cite}}. In all, 24 monitoring sites within the Puget Sound region (Fig. \colorbox{red}{\lstinline{1}}) were used in analyses. These represent 19 rivers across 9 counties, and range from 4 to 775 m in elevation. For one site at each river, monthly discharge time series were available, either for the same location as one of the temperature monitoring sites, or for another location within 30 km on the same major reach. Discharge data were aggregated by monthly mean from the USGS Washington Water Science Center (collected daily 1978-2007) and the USGS National Water Information System (collected at 15-minute intervals 2008-2015). At least one discharge monitoring site was available for every river represented in the temperature dataset.


Regional climate data were ... the Puget Sound Lowland and East Olympic/Cascade Foothills climate divisions

\end{document}



% https://wa.water.usgs.gov/data/realtime/adr/interactive/ #discharge daily
% https://waecy.maps.arcgis.com/apps/Viewer/index.html?appid=832e254169e640fba6e117780e137e7b #Q 15min
% http://www.horizon-systems.com/NHDPlus/NHDPlusV2_17.php #nhdplus
% ftp://newftp.epa.gov/EPADataCommons/ORD/NHDPlusLandscapeAttributes/StreamCat/HydroRegions/ #streamca
% http://www.ncdc.noaa.gov/cag/time-series/us/45/3/tavg/ytd/12/1895-2016?base_prd=true&firstbaseyear=1901&lastbaseyear=2000 #climate data
